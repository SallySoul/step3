\documentclass[9pt]{extarticle}


% Set margins
\usepackage[
  letterpaper,
  left=2cm,
  right=2cm,
  top=2cm,
  bottom=2cm
]{geometry}
\setlength{\parindent}{20pt}

\usepackage[none]{hyphenat}

% No page number
\pagenumbering{gobble}

%\lfoot{  testing}
\usepackage{fancyvrb}
\VerbatimFootnotes

% For multi-column environment
\usepackage{multicol}
\setlength{\columnsep}{1cm}

% Formatting for numbers with superscript
\usepackage[super]{nth}

% Allows us to have a footnote without a key
\newcommand\blfootnote[1]{%
  \begingroup
  \renewcommand\thefootnote{}\footnote{#1}%
  \addtocounter{footnote}{-1}%
  \endgroup
}

% Define a macro for full-width dot fill with a tab stop at column 2
\newcommand{\dotfillcols}[2]{%
  \noindent
  #1\leaders\hbox to .3em{\hss.\hss}\hfill
  \makebox[\dimexpr(\textwidth - \columnsep)/2\relax][l]{#2}\par
}

% Same as above but without dots
\newcommand{\nofillcols}[2]{%
  \noindent
  #1\leaders\hbox to .3em{\hss \ \hss}\hfill
  \makebox[\dimexpr(\textwidth - \columnsep)/2\relax][l]{#2}\par
}

% In lieu of using sections / subsections
\newcommand{\action}[1]{\noindent{\large \textbf{\textsf{#1}}}}
\newcommand{\actionspacing}[0]{\vspace{0.3cm}}


\begin{document}
\noindent{\huge \textbf{\textsf{Step Three - HOW to turn over your will}}}

\noindent\textit{Made a decision to turn our will and our lives over to the care of our Higher Power}

\vspace{0.3cm}
\hrule height 1pt
\vspace{-0.1cm}
\begin{multicols}{2}
\action{Take Actions That Deflate Your Ego}\\
\noindent 
Yes, the \nth{3} Step \textit{starts} with the phrase ``made a decision,''
but despite what some may say, it cannot end there.
Once you have made the decision, 
you must take \textbf{ACTIONS} to turn over your will, 
or that decision is meaningless.
(i.e. ``faith without works is dead'').

Turning over your will means reducing the influence of your ego.
To deflate your ego, you must practice 
\textbf{relinquishing control and letting go of stories}.

Quieting the influence of your ego will feel unnatural
and uncomfortable, and it will take conscious effort at first.
But over time, through \textbf{PRACTICE}, it will get easier, 
and eventually it will become second nature.

\actionspacing

\action{Ask for Help} \\
\noindent
However you found your way to your first meeting,
you either explicitly or implicitly asked for help.
This was the very first way you practiced turning over your will --
when you acknowledged that you couldn't do it alone.

Continue turning over your will by remembering 
that you don't have to face \textit{any} problem alone. 
If you have questions, ask them! 
If you need help, ask for it!
If you have importance decisions to make, 
solicit advice from other sober people.

\actionspacing

\action{Take Suggestions} \\
\noindent
After asking for help, 
the next step in turning over your will is to act on 
the advice offered by those with more experience.
Often the \textbf{ACTION of taking the suggestion} 
is as important as or more important than the text of the suggestion.
The power you take away from your ego will bring relief.

You may hear some crazy suggestions, 
but don't try to use those as evidence 
that taking suggestions is a bad idea in general;
try to stick to those suggestions you hear a lot.
Remember that shopping around for the piece 
of advice that you \textit{want} to hear 
-- to justify what you've already decided to do 
-- is just another form of alcoholic self-will.

Getting a sponsor formalized the process of taking suggestions,
providing you with a person whose suggestions you agreed to take especially seriously.
Sponsors will not abuse this privilege.
If you trust your sponsor and treat their advice very gravely, 
you will experience relief from you ego's reflexive desire to control everything.

%% NEXT COLUMN
\columnbreak

\action{Meditate} \\
\noindent
Don't wait until the \nth{11} step to start meditating.
For alcoholics, the greatest benefit of meditation 
is the ability to quiet the ego through focus on the body in the moment.
The ego tells stories, and stories require the past of the future;
focusing on the present there \textit{automatically} reduces the influence of the ego.
Focusing on the body draws attention away from our thoughts.

\actionspacing

\action{Practice Acceptance} \\
\noindent
Demanding that the world be a certain way 
-- or believing you would be happy if only some external conditions were met 
-- may be the most common form of self-will.
The truth is that you will not be happy until you learn to accept life on life's terms.

When faced with a situation that you find unpleasant,
try saying ``this, too, I can accept.'' 
It's difficult and requires practice to learn to employ this tactic.

{\small \textit{Grant me the Serenity to accept the things I cannot change}}

\actionspacing

\action{Practice Intention} \\
\noindent
We cannot live our lives passively.
Taking action requires intent,
and communication with your higher power.
This will look different for everyone.
We see manifesting, positive self talk, prayer, journaling, even magic
as expressions of will and self-actualization.
Reflect on your life, express gratitude, 
investigate what the next-right thing should be.

As members of AA we change many aspects of our life through step-work.
Often change is deeply desired.
And as humans we should grow and change.
However, the change must come from within.
We have to take the reigns and put in the work.

{\small \textit{Grant me the courage to change the things I can.}}

\actionspacing

\action{Do the Fourth Step} \\
\noindent
If you work a \textit{perfect} \nth{3} step -- 
completely turning over you will --
the rest of the steps wouldn't be necessary.
But you can't work a perfect \nth{3},
so keep trying and move forward! 
Think of step 3 as a decision to turn over your will
\textit{by doing the rest of the steps}

\end{multicols}

\vspace{0.4cm}
\hrule height 1pt
\vspace{0.3cm}

\nofillcols{\action{Symptoms of self-will}}{\action{Signs that you are turning your will over}}
\dotfillcols{You are getting upset.}{You are calm.}
\dotfillcols{You are telling endless stories in your head.}{You are not plagued by stories.}
\dotfillcols{You are justifying your actions -- to yourself or others.}{You feel no need to justify you actions.}
\dotfillcols{You're making a lot of plans.}{You are focused on the \textit{next} right thing.}
\dotfillcols{You think you can -- or should -- handle everything alone.}{You are open to help.}
\dotfillcols{You argue with or reject the suggestions of others.}{you can let go and take suggestions without fear.}

\begingroup
\renewcommand\thefootnote{}\footnote{\verb|step3_sbft_v1.pdf| testing}
\addtocounter{footnote}{-1}
\endgroup

\end{document}
