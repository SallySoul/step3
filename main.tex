\documentclass[9pt]{extarticle}

\pagenumbering{gobble}

\usepackage[
  letterpaper,
  left=2cm,
  right=2cm,
  top=2cm,
  bottom=2cm
]{geometry}

\usepackage{multicol}

\usepackage{array}

\setlength{\parindent}{20pt}


\usepackage{calc}

% Define a macro for full-width dot fill with a tab stop at column 2
\newcommand{\dotfillcols}[2]{%
  \noindent
  \makebox[0pt][l]{#1}% Left text flush left
  \leaders\hbox to .4em{\hss.\hss}\hfill
  \makebox[\dimexpr(\textwidth - \columnsep)/2\relax][l]{#2}\par
}


\setlength{\columnsep}{1cm}

% Define a precise dotfill macro aligned with multicol columns
\newcommand{\dotfillcolst}[2]{%
  \noindent
  % width of one column:
  \newlength{\colw}%
  \setlength{\colw}{\dimexpr(\textwidth - \columnsep)/2\relax}%
  %
  \makebox[0pt][l]{#1}% left text (natural width)
  \leaders\hbox to .4em{\hss.\hss}\hskip
    \dimexpr(\colw - \widthof{#1})\relax
  #2\par
}

\makeatletter
\newcommand{\dotfillcolstt}[2]{%
  % compute one column width
  \newlength{\colw}%
  \setlength{\colw}{\dimexpr(\textwidth - \columnsep)/2\relax}%
  \newlength{\colawidth}
  \setlength{\colawidth{\widthof{#1}}}
  %
  % measure #1 width
  \settowidth{\@tempdima}{#1}%
  %
  \noindent
  % print left text
  #1%
  % dotted fill to end of first column
  \leaders\hbox to .4em{\hss.\hss}\hskip
    \dimexpr(\colw - \@tempdima)\relax
  % second column text
  #2%
  \par
}
\makeatother


\newcommand{\action}[1]{\noindent{\large \textbf{\textsf{#1}}}}

\newcommand{\actionspacing}[0]{\vspace{0.3cm}}

\begin{document}

\noindent{\huge \textbf{\textsf{Step Three - HOW to turn over your will}}}

\noindent\textit{Made a decision to turn our will and our lives over to the care of God as we understood him}

\hrule

\begin{multicols}{2}

\action{Take Actions That Deflate Your Ego}\\
\noindent 
Yes, the $3^{rd}$ Step \textit{starts} with the phrase ``made a decision,''
but despite what some overly literal AAs will tell you, it cannot end there.
Once you have made the decision, 
you must take \textbf{ACTIONS} to turn over your will, 
or that decision is meaningless.
(i.e. ``faith without works is dead'').

Turning over your will means reducing the influence of your ego.
To deflate your ego, you must practice 
\textbf{relinquishing control and letting go of stories}.

Quieting the influence of your ego will feel unnatural
and uncomfortable, and it will take conscious effot at first.
But over time, through \textbf{PRACTICE}, it will get easier, 
and eventually it will become second nature.

\actionspacing

\action{Ask for Help} \\
\noindent
However you found your way to your first meeting,
you either explicitly or implicitly asked for help.
This was the very first way you practiced turning over your will --
when you acknowledged that you couldn't do it alone.

Continue turning over your will by remembering 
that you don't have to face \textit{any} problem alone. 
If you have questions, ask them! If you need help, ask for it!
If you have importance decisions to make, solicit advice from other sober people.

\actionspacing

\action{Take Suggestions} \\
\noindent
After asking for help, the next step in turning over your will is to act on the advice offered by those with more experience.
Often the \textbf{ACTION of taking the suggestion} is as important as or more important than the text of the suggestion.
The power you take away from your ego will bring relief.

You may hear some crazy suggestions, but don't try to use those as evidence that taking suggestions is a bad idea in general;
try to stick to those suggestions you hear a lot.
Remember that shopping around for the piece of advice that you \textit{want} to hear 
-- to justify what you've already decided to do -- is just another form of alchoholic self-will.

Getting a sponsor formalized the process of taking suggestions,
providing you with a person whose suggestions you agreed to take especially seriously.
Sponsors will not abuse this privilege.
If you trust your sponsor and treat his or her advice very gravely, you will experience relief from you ego's reflexive desire to control everything.

%% NEXT COLUMN
\columnbreak

\action{Pray} \\
\noindent
Whether or not you believe in antihistamines,
taking them will relieve allergy symptoms. 
Prayer is alot like medicine: you don't need to believe it's going to work, or understand how it's going to work, to get the benefits.

What, if anything, you pray to, is up to you;
there are as many different concepts of a higher power as there are people in AA. But the on thing that AAs have in common is the \textbf{ACTION} of praying.

I believe that the relief comes from the asking itself -- the explicit, verbal acknowledgment that reliance on your ego is not the answer. 
Your belief may differ, but one thing is clear: it doesn't come without the asking.

Here is a modernized, non-theist $3^{rd}$ step prayer:
\textbf{``Thank you for keeping me sober. 
Please keep me sober today. 
Relieve me of the bondage of self. 
Show me how I can be of service. 
Help me let go of my will.''}

\actionspacing

\action{Meditate} \\
\noindent
Don't wait until the $11^{th}$ step to start meditating.
For alcoholics, the greatest benefit of meditation is the ability to quiet the ego through focus on the body in the moment.
The ego tells stories, and stories require the past of the future;
focusing on the present there \textit{automatically} reduces the influence of the ego.
Focusing on the body draws attention away from our thoughts.

\actionspacing

\action{Practice Acceptance} \\
\noindent
Demanding that the world be a certain way --
or believing you would be happy if only some external conditions were met --
may be the most common form of self-will.
The truth is that you will not be happy until you learn to accept life life's terms.

When faced with a situation that you find unpleasant,
she recommends saying ``this, too, I can accept.'' 
It's difficult and requires practice to learn to employ this tactic.
I highly recommend reading Tara Brach's \textit{Radical Acceptance} when you start practicing your third step.

\actionspacing

\action{Do the Fourth Step} \\
\noindent
If you work a \textit{perfect} $3^{rd}$ step -- 
completely turning over you will --
the rest of the steps wouldn't be necessary.
But you can't work a perfect $3^{rd}$,
so keep trying and move forward! 
Think of step $3$ as a decision to turn over your will
\textit{by doing the rest of the steps}

\end{multicols}

\hrule

\dotfillcolstt{\action{Symptoms of self-will}}{\action{Signs that you are turning your will over}}

\begin{multicols}{2}
\subsection*{Symptoms of self-will}
You are getting upset
You are telling endless stories in your head \\
You are justifying your actions -- to yourself or others \\
You're making a lot of plans \\
You think you can -- or should -- handle everything alone \\
You argue with or reject the suggestions of others
\columnbreak
\subsection*{Signs that you are turning your will over}
You are calm. \\ 
You are not plagued by stories. \\
You feel no need to justify you actions. \\
You are focused on the \textit{next} right thing. \\
You are open to help. \\
you can let go and take suggestions without fear.
\end{multicols}

\end{document}
